\documentclass{article}
\usepackage{amsmath}
\usepackage{amssymb}
\usepackage{changepage}
\title{Problem Set 2}
\author{Michael Tang}
\begin{document}
\maketitle
\pagenumbering{arabic}
\begin{adjustwidth}{-2cm}{-2cm}

\section{Problem 1: Stock \& Watson}
\subsection{Exercise 3.2}
\textbf{(a)}\\
$\overline{Y} = E\left[Y_{1},...,Y_{n}\right] = \hat{p}*1 + \left(1-\hat{p}\right)*0$\\
$\overline{Y} = \hat{p}$\\
\textbf{(b)}\\
$E\left[\overline{Y}\right] = E\left[\hat{p}\right] = \frac{1}{n}\sum_{i=1}^{n}E\left(Y_{i}\right) = \frac{1}{n}\sum_{i=1}^{n}p = p$\\
Therefore, $\hat{p}$ is an unbiased estimator of $p$.\\
\textbf{(c)}\\
$var\left(\hat{p}\right) = var\left(\overline{Y}\right) = var\left(\frac{1}{n}\sum_{i=1}^{n}Y_{i}\right) = \frac{1}{n^{2}}\sum_{i=1}^{n}var\left(Y_{i}\right)$\\
$var\left(\hat{p}\right) = \frac{1}{n^{2}}\sum_{i=1}^{n}\sigma_{Y}^2$, $\sigma_{Y} = p\left(1-p\right)$ for Bernoulli rv.\\
$var\left(\hat{p}\right) = \frac{p\left(1-p\right)}{n}$  $\square$
\subsection{Exercise 3.3}
\textbf{(a)}\\
$\hat{p} = 215/400 = .538$\\
\textbf{(b)}\\
$SE\left(\hat{p}\right) = \sqrt{var\left(\hat{p}\right)} = \sqrt{\frac{\hat{p}\left(1-\hat{p}\right)}{n}} = \sqrt{\frac{.538*.462}{400}} = .025$\\
\textbf{(c)}\\
By the Central Limit Theorem, $\left\vert\frac{\hat{p}-\mu_{0}}{SE\left(\hat{p}\right)}\right\vert$ has approximate distribution $N\left(0,1\right)$.
\begin{align*}
p-value &= P_{H_{0}}\left\{\left\vert \frac{\hat{p}-\mu_{0}}{SE\left(\hat{p}\right)}\right\vert > \left\vert\frac{\hat{p}^{act}-\mu_{0}}{SE\left(\hat{p}^{act}\right)}\right\vert\right\}\\
&= 2\Phi\left(-\left\vert \frac{.538-.5}{.025} \right\vert\right) = 2\Phi\left(-1.52\right) = 2\left(.0643\right)\\
&= .1286
\end{align*}
\textbf{(d)}\\
\begin{align*}
p-value &= P_{H_{0}}\left\{\frac{\hat{p}-\mu_{0}}{SE\left(\hat{p}\right)} > \frac{\hat{p}^{act}-\mu_{0}}{SE\left(\hat{p}^{act}\right)}\right\}\\
&= \Phi\left(\frac{.038}{.025}\right) = \Phi\left(-1.52\right)\\
&= .0643
\end{align*}
\textbf{(e)}\\
p-value of (d) is one half that of (c) because it only counts one direction of inequality and not two.\\
\textbf{(f)}\\
Depends on the significance level one wants to take. Assuming a significance level $\alpha = .05$, both p-values show that we do not have significant enough evidence to suggest that the incumbent is ahead of the challenger.
\subsection{Exercise 3.4}
\textbf{(a)}\\
$95\%$ confidence for $p = \left\{\hat{p} \pm 1.96*SE\left(\hat{p}\right)\right\} = \left\{.538 \pm .049\right\} = \left\{.489, .587\right\}$\\
\textbf{(b)}\\
$99\%$ confidence for $p = \left\{.538 \pm 2.58*.025\right\} = \left\{.4735, .6025\right\}$\\
\textbf{(c)}\\
The interval in (b) is wider because to be able to capture the population fraction in a larger percentage of samples one has to expand the accepted range of values around the sample mean.\\
\textbf{(d)}\\
We cannot reject the null hypothesis at the $5\%$ significance level; see part (f) of Exercise 3.3.
\subsection{Exercise 3.15}

\subsection{Exercise 3.16}
\subsection{Exercise 3.17}
\subsection{Exercise 3.18}

\section{Problem 2}
\textbf{(a)}\\
\textbf{(b)}\\
\textbf{(c)}\\
\textbf{(d)}\\
\textbf{(e)}\\
\textbf{(f)}\\

\section{Problem 3}
\textbf{(a)}\\
\textbf{(b)}\\
\textbf{(c)}\\

\section{Problem 4}
\textbf{(a)}\\
\textbf{(b)}\\
\textbf{(c)}\\

\section{Problem 5}
\textbf{(a)}\\
\textbf{(b)}\\

\end{adjustwidth}
\end{document}